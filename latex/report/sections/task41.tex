\chapter{Public transport in large cities worldwide}

\resp{Quaglio Emanuele}

\section{Dataset and framework}\label{dataset-and-framework}

The dataset at \cite{citylines.co} report the
development over time of public transportation networks (different modes
and potentially different lines per mode) across a variety of cities
around the world. I build for each city a node list (nodeID, nodeLabel, latitude, longitude, mode, year) and and edges list (nodeID\_from, nodeID\_to, mode, line, year). The data is essentially
tabular, with entities Stations and Sections roughly corresponding to
nodes and edges, plus additional tables for transportation lines,
systems, and modes, and tables storing relationships between these
entities. The data is geometric in nature: each station features a
position, and each section features a polygonal chain. Position and
chain are represented respectively as Point and LineString according to
GeoJSON standard \cite{geoJSON}, based on \verb|(longitude, latitude)| coordinates following
WGS 84 system. The schemas of the tables are the following: 
\begin{itemize}
 \item stations:
id, name, geometry, buildstart, opening, closure, city\_id;
 \item sections:
id, geometry, buildstart, opening, closure, length, city\_id;
 \item cities:
id, name, coords, start\_year, url\_name, country, country\_state,
length; 
 \item lines: id, city\_id, name, url\_name, color, system\_id, transport\_mode\_id;
 \item systems: id, city\_id, name, historic, length; -
transport\_modes: id, name; 
 \item section\_lines: id, section\_id, line\_id,
created\_at, updated\_at, city\_id, fromyear, toyear,
deprecated\_line\_group; 
 \item station\_lines: id, station\_id, line\_id,
city\_id, created\_at, updated\_at, fromyear, toyear,
deprecated\_line\_group.
\end{itemize}



\section{Pipeline and preliminary
analysis}\label{pipeline-and-preliminary-analysis}

\subsection{From raw JSON to specified node/edge
files.}\label{from-raw-json-to-specified-nodeedge-files.}

\paragraph{Inspection and conversion.} 
From an inspection of the JSON data
files, I confirm they have columnar structure, so I decide to convert
them into a more convenient tabular representation. I choose Parquet
over CSV for type-preservation (since they are present in the original
JSON) and higher speed / memory efficiency.

\paragraph{Graph construction.} After filtering table rows by city, I
define the desired fields for the nodes and edges files, skipping cities
with no stations. 

\paragraph{} For the nodes: 
\begin{itemize}
\item nodeLabel: use \texttt{name} field or
(if not available) fallback to original ID; 
\item nodeID: sort by original
IDs and remap to dense indexes; 
\item year: use \texttt{Stations.opening} of
the station or (if not available) fallback to its earliest
\texttt{station\_lines.fromyear}; 
\item mode: in order not to lose or
distort mode information, I preserve all the (sorted, unique) modes
associated by joining \texttt{station\_lines}-\textgreater{}\texttt{lines}
-\textgreater{}\texttt{transport\_modes}, concatenated. The idea is that
for mixing analysis it'll be possible to convert it into a custom
\texttt{multiple-modes} mode feature, to properly count edges that
connect stations ``heterogeneously''. But at the same time, it will be
still possible to do finer analysis (checking presence of a specific
mode in the set obtained from splitting by \texttt{"\textbar{}"}
separator).
\end{itemize}

\paragraph{} For the edges, I decide to \textbf{admit multigraphs} (and simplify them when necessary during analysis). Indeed, I identify them by both keys
\texttt{section} and \texttt{line}. The motivation behind this choice is
that I think it would be useful to be able to identify a station as a
``hub'' when multiple lines depart from it, even when they share the
same physical section (e.g. shared rails).
\begin{itemize}
\item line: use \verb|Lines.url_name| or fallback to \verb|Lines.name|; 
\item year: use \verb|section_lines.fromyear| or fallback to \verb|Sections.opening|; 
\item \verb|nodeID_from|, \verb|nodeID_to|: associate each
section endpoint to a node based on the geometric procedure described
below; furthermore, sort endpoint IDs and drop self-loops.
\end{itemize}

\paragraph{Edge-to-nodes snapping procedure.} 
Each endpoint of a \texttt{section} is snapped to the \texttt{station} at minimal Haversine distance. To avoid spurious links due to noisy data, a section is kept
only as long as said minimal distances are below a certain
city-dependend threshold: 
\begin{equation}
thr_{city}=\min(\max(200m,\ d_{q99}),\ 1500m)  
\end{equation}
where $d_{q99}$ is the 99th percentile distance between endpoint and nearest station, distribution computed over all city section endpoints.
The acceptance threshold is higher for cities where said minimal distances are generally higher (due to noisy data acquisition / more sparse transport system / etc\ldots), but can never be below 200m or above 1500m.

\subsection{Sanity checks, filtering, global
analysis.}\label{sanity-checks-filtering-global-analysis.}

\paragraph{Preliminary global analysis.} To check the quality of the
produced networks I follow these steps: 
\begin{enumerate}
\item I build per-city networks with \texttt{igraph}: both muligraphs and \emph{simplified} graphs. 
\item I compute, per each city/network (mostly on the simplified graph), some
relevant summary metrics, stored in \texttt{per\_city\_summary.csv}: number of nodes, number of edges of the simple
    and the multigraph, density of the simple graph; mean degree of the
    simple and the multigraph; number
    of connected components, giant component fraction, global clustering
    coefficient; mean, median, and 90th percentile distance, and diameter, of the LCC; number of communities and resulting modularity
    after applying Louvain and Infomap method; mean and
    max edge multiplicity of the multigraph.
\item I generate and visualize across-cities
distributions of said metrics. 
\item To better inspect each global metrics
and the underlying individual networks, I select some representative
cities: in particular, those approximately located at the mean value, at
the 10th percentile and at the 90th percentile of the across-cities
distribution. This is done in order to have a glimpse of both the most
typical and some more atypical networks, still neglecting fully
pathological reconstructions. 
\item Finally, I plot the simple and
multigraph and the degree distribution of these representative cities.
\end{enumerate}

\paragraph{Pathology filtering.} Upon inspection, aided by the previous procedure, I observe that a
substantial proportion of cities present a patological network
reconstruction: the most common phenomenon is presence of very long
edges connecting non-subsequent stations, in some cases just the first
and last of a line. I suspect this could be caused by incomplete data on
the sections \texttt{LineString}, by sections that weren't updated upon
later introduction of new stops and stations, or simply by noisy
geometric data. To partially amend this problem and reconstruct cleaner
(although possibly biased) global statistics, I attempt filtering out
this family of pathological networks by discarding networks with a mean
degree lower than a certain threshold (since typical transport lines are
made of linear chains of stops connected by hub stations). By setting
this threshold to 1, more than 20\% of the cities are kept for further
analysis, whose networks appear reasonably well reconstructed upon
visual inspection.

\begin{figure}
    \centering
    \includegraphics[width=0.49\linewidth]{figures/task41/n_nodes_distribution.pdf}\hfill
    \includegraphics[width=0.49\linewidth]{figures/task41/reference_cities_simple_graphs.pdf}
    \caption{Left: size distribution of public transport networks across several cities in the world. The cities are filtered by mean degree $\geq 1$. Right: the reconstructed networks of Genoa, Vienna and Bordeaux, respectively representing the 10th percentile, mean, and 90th percentile of the global size distribution. Node size encodes degree, node color encodes community according to Louvain method.}
    \label{fig:n_nodes_and_representatives}
\end{figure}


\newpage